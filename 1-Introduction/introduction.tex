\chapter{Introduction}
\label{ch:Introduction}

The INTERLACE project is developing a blockchain-based transactional platform for use by the Sardex mutual credit system. Sardex S.p.A. (SARDEX) has been operating successfully an electronic, B2B, zero-interest mutual credit system on the island of Sardinia since 2009. The Sardex system (also known as Circuito di Credito Commerciale) challenges prevailing notions about the nature of money, and the financial and economic autonomy that a relatively poor region can aspire to, because it enables local economic actors (SMEs in particular) to trade with each other in a trustful and circular fashion with a unique digital trade credit unit. It does this by monetizing the spare capacity of the local economy in the form of mutual, and taxable, credit between participating companies, at zero interest, on a strong basis of trust, solidarity, and local cultural identity \cite{LitteraEtAl2017,DiniMottaSartori2016,SartoriDini2016}. The deeply innovative nature of this system is to distribute to the circuit members the power to create credit money (sardex credits, where 1 credit = 1 Euro), and thereby provide an alternative to credit money creation through bank loans. However, the Sardex transactional platform is currently centralized,\footnote{Cyclos 4: \url{http://www.cyclos.org/products/}.} which challenges the long-term scalability, sustainability, and governance of the system because the governance and management of the circuit are all held by the central credit-clearing entity (SARDEX).

The specification and implementation of the new architecture is based on the Abstract State Interaction Machines (ASIMs), developed and implemented in the platform CoreASIM by the BIOMICS project\footnote{\url{http://biomicsproject.eu/}} \cite{BIOMICSD41,BIOMICSD42,BIOMICSD52} as extensions of B\"orger and St\"ark's \cite{BoergerStaerk2003} Abstract State Machines (ASMs) and of the CoreASM environment.\footnote{\url{http://biomicsproject.eu/news/135-icef}} The ASIMs are based on realizing the BIOMICS mathematical framework for Interaction Machines (IMs) that dynamically and recursively grow and change their components and network topologies to deploy/reabsorb resources in response to interactions and computational needs \cite{NehanivEtAl2015}. Relative to the ASMs, ASIMs are fully asynchronous, concurrent, and communicating, so they can run on different servers and communicate over the network to validate transactions. The ASIM approach is fundamentally important for SARDEX as a company because it guarantees verifiability, validation, and efficient change management (to manage requirements creep as well as new emerging functionalities) through rigorous mathematical formalization of the specification at the level of requirements capture and a rigorous process of iterative refinement down to the level of the code of choice.
Any of these levels of abstraction is executable by an interpreter (built into coreASIM), so at each level of the refinement process the current implementation level can be verified against requirements.

The INTERLACE proposal describes a blockchain architecture based on the Open Transaction protocol (OTX) \cite{Odom} as an intermediate solution between fully centralized and distributed architectures. OTX involves a pool of Auditor nodes to validate the transactions executed by each Notary node. The initial conception of the INTERLACE architecture was to use only one central Notary, as a first step from the current centralized server towards a more distributed architecture. For the persistence layer we originally examined the Lightweight Cryptocurrency Ledger (LCL) \cite{White2015} as an alternative to the Bitcoin blockchain. The idea of LCL is to store minimal information on the current state of the whole ledger in each block rather than requiring nodes to carry the whole history of block deltas as the Bitcoin blockchain does. This approach achieves continuity with the existing solution while also enabling scalability to multiple circuits (multiple Notaries) under the same mathematical and computational framework.

Rather than replicating the functionality of Cyclos 4 in a decentralized or distributed manner, by implementing the OTX-LCL concepts from scratch, it makes sense to take advantage of the staggering levels of investment currently being made in blockchain technologies by several banking and industry consortia, especially since the best ones are open source, and build on existing frameworks. As of April 2017, just before INTERLACE started, the two most likely candidates for our purposes were IBM's Hyperledger Fabric\footnote{\url{https://hyperledger-fabric.readthedocs.io/en/latest/}}\cite{Cachin2016} and Corda.\footnote{\url{https://www.corda.net/}} Together, they can be described as bringing together the principles of OTX and LCL with the smart contracts of Ethereum.\footnote{\url{https://www.ethereum.org/}} However, an important difference is that these blockchain implementations are �permissioned�, they are not open (�permissionless�) like the Bitcoin or Ethereum blockchains. Although this suits the early decentralised implementation of the INTERLACE blockchain, it opens up governance questions that are currently being examined by Sardex S.p.A. First among these question is the possibility of establishing two separate legal entities:
\begin{packed_item1}
\item Sardex S.p.A. (SARDEX) will maintain its current ownership structure.
\item A new non-profit legal entity, which we can refer to for now as Circuit Coop, or more loosely as sardex.net, may be formed in the near future to begin devolving the ownership of the commons\footnote{The Sardex circuit commons have not been defined yet, but they will constitute a new basis for shared ownership by all the members. An example could be solar panels owned by the coop and providing renewable energy that can only be bought with credits.} to the circuit members.
\end{packed_item1}

From a purely functional point of view, Hyperledger enables a given node to belong to multiple blockchain networks. This is of interest to us because in the governance framework SARDEX is defining at least two units to be stored on the blockchain are envisaged: SRD credits and a new meritocratic reward points system called `Proximity', implemented with tokens that are earned through altruistic behaviour and dubbed `$\pi$'. Corda is interesting because it separates what Hyperledger calls `chaincode', which implements the smart contracts, from the persistence layer. Thus, Corda comes with a `business flow' layer that provides greater flexibility to adapt to the great diversity of actors in the four partially overlapping networks and across different regions or even countries.

The number of new cryptocurrencies is increasing very rapidly,\footnote{As of 28/08/17 there were 865 currencies listed on \url{https://coinmarketcap.com/}, up from 851 a few days earlier.} along with the variations in the technologies that support them. This very volatile technology landscape is causing us to focus first on the new economic and governance model for sardex.net, which will provide the high-level requirements for the new blockchain architecture and will, therefore, enable us to narrow down the number of possible frameworks and technologies to draw from. In the meantime, we have begun the non-trivial task of migrating the current platform functionality towards the new model. The first step in this process has been to begin formalizing the SARDEX business logic as ASM models.

This report is organized as follows. Chapter \ref{ch:archreq} provides a high-level view of the architecture and its documentation following the arc42 method.\footnote{\url{http://arc42.org/}} Chapter \ref{ch:funreq} provides a first collection of functional requirements of the system modelled as ASMs. These reflect the business logic of the \emph{current} system. This model will be instantiated in CoreASIM in order to execute it with a fictitious set of inputs and verify its operation, after which it will be implemented in a language of choice (probably Java).

A second specification and modelling effort will follow, to reflect the new governance and financial/economic model once it has been completed. This second model and its implementation will be reported in the next architecture deliverable (D2.2) at Month 12.











