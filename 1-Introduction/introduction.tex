\chapter{Introduction}
\label{ch:Introduction}

\vspace{-1cm}
\begin{center}
Paolo Dini and Giuseppe Littera
\end{center}

The INTERLACE project is developing a blockchain-based transactional platform for use by the Sardex mutual credit system. Sardex S.p.A. (SARDEX) has been operating successfully an electronic, B2B, zero-interest mutual credit system on the island of Sardinia since 2009. The Sardex system (also known as Circuito di Credito Commerciale) challenges prevailing notions about the nature of money, and the financial and economic autonomy that a relatively poor region can aspire to, because it enables local economic actors (SMEs in particular) to trade with each other in a trustful and circular fashion with a unique digital trade credit unit. It does this by monetizing the spare capacity of the local economy in the form of mutual, and taxable, credit between participating companies, at zero interest, on a strong basis of trust, solidarity, and local cultural identity \cite{LitteraEtAl2017,DiniMottaSartori2016,SartoriDini2016}. The deeply innovative nature of this system is to distribute to the circuit members the power to create credit money (sardex credits, where 1 credit = 1 Euro), and thereby provide an alternative to credit money creation through bank loans. However, the Sardex transactional platform is currently centralized,\footnote{Cyclos 4: \url{http://www.cyclos.org/products/}.} which challenges the long-term scalability, sustainability, and governance of the system because the governance and management of the circuit are all held by the central credit-clearing entity (SARDEX).

The specification and implementation of the new architecture are based on the Abstract State Interaction Machines (ASIMs), developed and implemented in the platform CoreASIM by the BIOMICS project\footnote{\url{http://biomicsproject.eu/}} \cite{BIOMICSD41,BIOMICSD42,BIOMICSD52} as extensions of B\"orger and St\"ark's \cite{BoergerStaerk2003} Abstract State Machines (ASMs) and of the CoreASM environment.\footnote{\url{http://biomicsproject.eu/news/135-icef}} The ASIMs are based on realizing the BIOMICS mathematical framework for Interaction Machines (IMs) that dynamically and recursively grow and change their components and network topologies to deploy/reabsorb resources in response to interactions and computational needs \cite{NehanivEtAl2015}. Relative to the ASMs, ASIMs are fully asynchronous, concurrent, and communicating, so they can run on different servers and communicate over the network to validate transactions. The ASIM approach is fundamentally important for SARDEX as a company because it guarantees verifiability, validation, and efficient change management (to manage requirements creep as well as new emerging functionalities) through rigorous mathematical formalization of the specification at the level of requirements capture and a rigorous process of iterative refinement down to the level of the code of choice. Any of these levels of abstraction is executable by an interpreter (built into CoreASIM), so at each level of the refinement process the current implementation level can be verified against requirements.

The number of new cryptocurrencies is increasing very rapidly,\footnote{As of 28/08/17 there were 865 currencies listed on \url{https://coinmarketcap.com/}, up from 851 a few days earlier.} along with the variations in the technologies that support them. This very volatile technology landscape is causing us to focus first on the new economic and governance model for sardex.net, which will provide the high-level requirements for the new blockchain architecture and will, therefore, enable us to narrow down the number of possible frameworks and technologies to draw from. In the meantime, we have begun the non-trivial task of migrating the current platform functionality towards the new model. The first step in this process has been to begin formalizing the SARDEX business logic as ASM models.

This report is organized as follows. Chapter \ref{ch:archreq} provides a high-level view of the architecture and its documentation following the arc42 method.\footnote{\url{http://arc42.org/}} Chapter \ref{ch:funreq} provides a first collection of functional requirements of the system modelled as ASMs. These reflect the business logic of the \emph{current} system. This model will be instantiated in CoreASIM in order to execute it with a fictitious set of inputs and verify its operation, after which it will be implemented in a language of choice (probably Java).

A second specification and modelling effort will follow, to reflect in the blockchain ``backend'' the new governance and financial/economic model once it has been completed. This second model and its implementation will be reported in the next architecture deliverable (D2.2) at Month 12.











