\chapter{Introduction}
\label{ch:Introduction}


SARDEX has been operating successfully an electronic, B2B, zero-interest mutual credit system on the island of Sardinia since 2009. The Sardex system (also known as Circuito di Credito Commerciale) challenges prevailing notions about the nature of money, and the financial and economic autonomy that a relatively poor region can aspire to, because it enables local economic actors (SMEs in particular) to trade with each other in a trustful and circular fashion with a unique digital trade credit unit. It does this by monetizing the spare capacity of the local economy in the form of mutual, and taxable, credit between participating companies, at zero interest, on a strong basis of trust, solidarity, and local cultural identity \cite{DiniMottaSartori2016,SartoriDini2016}. The deeply innovative nature of this system is to distribute to the circuit members the power to create credit money (sardex credits, where 1 credit = 1 Euro), and thereby provide an alternative to credit money creation through bank loans. However, the Sardex transactional platform is currently centralized,\footnote{Cyclos 4: http://www.cyclos.org/products/.} which challenges the long-term scalability, sustainability, and governance of the system because the governance and management of the circuit are all held by the central credit-clearing entity (SARDEX).






SARDEX will use the Open Transaction protocol (OTX) as an intermediate solution between fully centralized and distributed architectures. OTX involves a pool of Auditor nodes to validate the transactions executed by each Notary node. In INTERLACE there will be only one central Notary, as a first step from the current centralized server towards a more distributed architecture. The persistence layer will be implemented as a private blockchain stored on the central server to create a sparse 160-bit address space implemented as a binary hash tree. This approach achieves continuity with the
existing solution while also enabling scalability to multiple circuits (multiple Notaries) under the same mathematical and computational framework. SARDEX has been operating successfully an electronic, B2B, zero-interest mutual credit system on the island of Sardinia since 2009. The Sardex system (also known as Circuito di Credito Commerciale) enables local economic actors (SMEs in particular) to trade with each other in a trustful and circular fashion with a unique digital trade credit unit. It does this by monetizing the spare capacity of the local economy in the form of mutual, and taxable, credit between participating companies, at zero interest, on a strong basis of trust, solidarity, and local cultural identity. Therefore, INTERLACE will address very effectively the Workprogramme objective to generate socio-economic impact from EU-funded research. INTERLACE is uniquely positioned to integrate the very advanced results of BIOMICS directly in a deeply innovative, transformative, and successful fintech platform for B2B trade.







\cite{BIOMICSD41,BIOMICSD42,BIOMICSD52} 

 a model based on Abstract State Machines (ASMs) \cite{BoergerStaerk2003} 









